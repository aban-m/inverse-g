
\newcommand{\sqf}[2]{#1_1 #1_2 \mathellipsis #1_{#2}}
\newcommand{\qs}{\sqf{q}{s}}
\newcommand{\ps}{\sqf{p}{s}}
\newcommand{\rs}{\sqf{r}{s}}
\newcommand{\ufd}{p_1^{\alpha_1} p_2^{\alpha_2} \cdots p_s^{\alpha_s}}
\newcommand{\ufdsh}{p_1^{\alpha_1} \cdots p_s^{\alpha_s}}

\newcommand{\nsub}{\triangleleft}
\newcommand{\fbr}[1]{g^{-1}{(#1)}}

\newcommand{\aut}[1]{\operatorname{Aut}(#1)}
\newcommand{\cyc}[1]{\operatorname{C}_{#1}}
\newcommand{\gl}[2]{\operatorname{GL}(#1, \mathbb{F}_#2)}
\newcommand{\Mod}[1]{\ (\mathrm{mod} \ #1)}

\newcommand{\qo}{\tilde{q}}
\newcommand{\pin}{\tilde{p}}
\newcommand{\tilpi}{\tilde{\pi}}
\newcommand{\piga}{\pi_\Gamma}
\newcommand{\pila}{\pi_\Lambda}
\newcommand{\fibo}[1]{F_{#1}}

\newcommand{\qlame}{\Lambda_{5,\text{I}}}
\newcommand{\qlamz}{\Lambda_{5,\text{II}}}
\newcommand{\qlamd}{\Lambda_{5,\text{III}}}
\newcommand{\slame}{\Lambda_{6, \text{I}}}
\newcommand{\slamz}{\Lambda_{6, \text{II}}}
\newcommand{\m}[1]{\text{M}_{#1}}
\newcommand{\hlame}{\Lambda_7}
\newcommand{\hthref}[1]{\hyperref[#1]{\thref{#1}}}

\NewDocumentCommand{\listspace}{}{}
\NewDocumentCommand{\textspace}{}{\linespread{1}\selectfont}
\NewDocumentCommand{\eqnspace}{}{\vspace{-0.0\baselineskip}}
\NewDocumentCommand{\drawunspace}{}{\vspace{-0.5\baselineskip}}
\NewDocumentCommand{\enumunspace}{}{}
\NewDocumentCommand{\unpre}{}{\hspace*{-8pt}}

\NewDocumentCommand{\case}{m}{\smash{\sffamily \textbf{Case #1.} \rmfamily }}
\NewDocumentCommand{\scase}{m}{\smash{\sffamily {Subcase #1.} \rmfamily }}

\NewDocumentCommand{\hold}{m}{Hölder}
\NewDocumentCommand{\hg}{m}{Hölder graph}
\NewDocumentCommand{\hgs}{m}{Hölder graphs}


\DeclarePairedDelimiter\abs{\lvert}{\rvert}


\theoremstyle{plain}
\newtheorem{thm}{Theorem}[section]
\newtheorem{eufact}{Fact}[section]
\newtheorem{lem}{Lemma}[section]
\newtheorem{prop}{Proposition}[section]
\newtheorem*{quest}{Question}
\newtheorem{cor}[prop]{Corollary}
\newtheorem*{conj}{Conjecture}

\theoremstyle{definition}
\newtheorem*{defn}{Definition}

\setenumerate[1]{label=\alph*.}
\setenumerate[2]{label=\alph*.}


% Fixing arXiv URL not-wrapping
\usepackage[normalem]{ulem}
\usepackage{url}
\renewcommand{\UrlBigBreaks}{}
\renewcommand{\UrlBreaks}{\do\.{\mathchar`\.}%
                            \do\/{\mathchar`\/}%
                            \do\@{\mathchar`\@}%
                            \do\\{\mathchar`\\}%
                            \do\-{\mathchar`\-}%
                            \do\#{\mathchar`\#}}

% Fixing a pagebreak towards the end 
% https://tex.stackexchange.com/questions/94699/absolutely-definitely-preventing-page-break
\newenvironment{keepintact}
  {\par\nobreak\vfil\penalty0\vfilneg
   \vtop\bgroup}
  {\par\xdef\tpd{\the\prevdepth}\egroup
   \prevdepth=\tpd}

\tikzset
{%
point/.style={fill=black, draw, circle, inner sep=0.042cm, outer sep=0.12cm},
soint/.style={fill=black, draw, circle, inner sep=0.022cm, outer sep=0.1cm},
walk/.pic={
	\node [soint] (a) {};
	\node [soint,right=of a] (b) {};
	\path [->] (a) edge (b);
},
twalk/.pic={
	\node [soint] (a) {};
	\node [soint,right=of a] (b) {};
	\node [soint, right=of b] (c) {};
	\path [->] (a) edge (b)
	(b) edge (c);
},
ewalk/.pic={
	\node [soint] (a) {};
	\node [soint, right=of a] (b) {};
},
w/.style={baseline=-0.4ex},
ww/.style={baseline=-0.4ex, node distance=0.37cm}
}


