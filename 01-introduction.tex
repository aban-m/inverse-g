The function $g(n)$, defined as the number of groups of order $n$ (up to isomorphism), has various interesting arithmetical properties. It is an elementary fact, for example, that we have $g(p) = 1$ for all primes $p$, and an application of Sylow theorems shows that $g(pq) = 1$ whenever $p, q$ are primes with $q \not\equiv 1 \Mod{p}$ and $q > p$. This raises the question: Which $n$ satisfy $g(n) = 1$? Such $n$ are called \emph{cyclic} numbers. It turns out that $n$ is cyclic precisely when it is coprime with $\phi(n)$, where $\phi$ is Euler's totient function {\cite{szele}}. More generally, we can consider the equation $g(n) = k$ for a fixed integer $k$.

Miller answers this question for $k = 1, 2, 3$ in {\cite{miller1}} and for $k = 4, 5$ in {\cite{miller2}}. It has been re-derived multiple times: For instance, a short derivation of the cases $1 \le k \le 3$ appears in {\cite{olsson}}, and for $1 \le k \le 4$ in {\cite{gnumoas}}. In this paper, we consider the cases $k = 6$ and $k = 7$. After this paper was written, I was made aware of a 1936 paper by D. T. {\cite{sigley}} addressing the case \mbox{$k = 6$.}

A beautiful formula for $g(n)$ when $n$ is square-free has been discovered by \hold{1} {\cite[Thm.~5.1]{gnumoas}}. An application of it is that, if $p, q, r$ are primes with $q \equiv r \equiv 1 \Mod{p}$ and $qr$ is a cyclic number, then $g(pqr) = p + 2$, already showing that the image of $g$ is infinite. There is no similarly explicit formula for cube-free $n$, but there is an efficient algorithm {\cite{cube-free}}, and there are relatively simple formulae for $g(n)$ when $n$ has a small number of prime factors with small exponents.

By forming direct products, it is clear that $g(ab) \ge g(a)g(b)$, at least when $a$ and $b$ are coprime. Since \mbox{$g(p^2) = 2$,} $g(p^3) = 5$ and $g(p^\alpha) \ge 14$ for all primes $p$ and all $\alpha \ge 4$ (see, for example, \cite[Thm.~3.1]{gnumoas}), it is enough to consider fourth-power-free $n$. The key computational tool will be directed graphs: We associate a directed graph to each number, with the vertices being the prime powers that divide $n$ and the edges indicating the existence of nontrivial semidirect products.
