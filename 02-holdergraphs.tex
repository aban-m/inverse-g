\subsection{Hölder's formula}
Given a square-free integer $n$, a prime factor $p$ and a subset $\pi$ of its prime factors, we let $v(p, \pi)$ be the number of $q$ in $\pi$ such that $q \equiv 1 \Mod{p}$.
In this case, we say that $p$ is \textbf{related} to $q$, and $p$ and $q$ are related.
We also let $\Gamma$ be the set of all prime factors of $n$.
\mbox{Then we have}
\begin{thm}[\textbf{\hold{1}'s formula}]\thlabel{euholder} In this notation,
	\begin{equation*}
		g(n) = \sum_{\pi \subseteq \Gamma} \prod_{p \notin \pi} \frac{p^{v(p, \pi)} - 1}{p - 1}
	\end{equation*}
\end{thm}
\begin{proof} See {\cite[Thm.~5.1]{gnumoas}}.
\end{proof}

It is therefore natural to define the \emph{\hg{1}} $\Gamma = \Gamma(n)$ of a square-free integer $n$ to be the unlabeled directed graph whose vertices are the prime factors, and where we have an edge $p \rightarrow q$ precisely when $p$ is related to $q$.
Observe also that when the out-degree of every vertex is at most 1, $g(n)$ depends solely on the shape of the \hg{1}, and therefore we are justified in speaking of $g(\Gamma)$.
We call both $n$ and $\Gamma$ \emph{regular} in this case.
We will occasionally treat directed acyclic graphs $\Gamma$ where we only know the \emph{labels}, or the numerical values, of the vertices whose out-degree exceeds $1$.
In this case, we can still speak of $g(\Gamma)$ without ambiguity, as it will be independent of the labels of the rest of the vertices.
We also observe that Dirichlet's theorem on arithmetic progressions shows that a given (unlabeled) directed acyclic graph is always realizable as the \hg{1} of some odd square-free integer.

We generalize this to arbitrary $n = \ufdsh$ as follows: The vertices are now the $p_i^{\alpha_i}$, and an edge $p^\alpha \rightarrow q^\beta$ exists precisely when $q^j \equiv 1 \Mod{p^i}$ for some $1 \le i \le \alpha$ and $1 \le j \le \beta$.
We will call this the \emph{generalized \hg} of $n$, also denoted by $\Gamma(n)$.
Typically, $n$ will be cube-free, in which case we write $p^2 \dashrightarrow q$ to mean $p \parallel q - 1$, and $q \dashrightarrow p^2$ to mean $q \mid p + 1$ but $q \nmid p - 1$.
We call these two types of arrows \emph{weak}, and other arrows \emph{strong}.

A vertex is said to be \emph{initial} if it has an out-degree of $0$, and \emph{terminal} if it has an in-degree of $0$.
We also let $S(\pi, \Gamma) = \prod_{p \notin \pi}h_\Gamma(p, \pi)$ be the summand in Hölder's formula corresponding to the subset $\pi$.
If there is a vertex $p$ not connected to any vertex in $\pi$, the entire summand $S(\pi \Gamma)$ vanishes.
Motivated by this observation, we call a subset $\pi$ \emph{central} if, for every vertex outside of it, there exists at least one edge to a vertex inside it -- equivalently, if $S(\pi \Gamma)$ does not vanish.
We remark that $n$ is regular precisely when $g(\Gamma(n))$ is equal to the number of central subsets.

\subsection{Connectivity and multiplicativity}
We have already observed that $g(ab) \ge g(a)g(b)$ for coprime $a, b$.
When does equality hold? If it does hold, then every group of order $ab$ splits as a direct product of groups of orders $a$ and $b$.
Coprimality is therefore a necessary condition: If $p$ divides both $a$ and $b,$ we simply observe that the group $\cyc{a/p} \times \cyc{p^2} \times \cyc{b/p}$ does not split in this way.
Next, it must be impossible to form semidirect products.
If $p$ divides $b$ and $q^n$ divides $a$, we observe that $\abs{\aut{\cyc{q}^n}}$ has $q^i - 1$ as a factor for all $1 \le i \le n$, and therefore a nontrivial semidirect product exists if there is a relation $p \rightarrow q^n$.
In this case, $a$ and $b$ are said to be \emph{arithmetically dependent}.
We conclude that another necessary condition is $\Gamma(n) = \Gamma(a) \sqcup \Gamma(b)$.
 The converse is also true:

\begin{thm}\thlabel{eufrobenius}
	Suppose we have $n = ab$.
Then the following are equivalent:
	\begin{enumerate}
		\item $g(n) = g(a)g(b)$,
		\item $\abs{G} = n$ implies $G \cong A \times B$ where $\abs{A} = a$ and $\abs{B} = b$,
		\item $a$ and $b$ are arithmetically independent,
		\item $\Gamma(n) = \Gamma(a) \sqcup \Gamma(b)$.
	\end{enumerate}
\end{thm}
\begin{proof}
	See {\cite[Lem.~21.19]{monolith}}.
\end{proof}

\begin{prop}\thlabel{euufd}
	For positive $n$, we have a unique factorization $n = n_1 \cdots n_k m$ where\pagebreak[3]
	\begin{enumerate} \listspace
		\item The $k + 1$ factors are pairwise arithmetically independent,
		\item Each $n_i$ is connected with $g(n_i) \ge 2$,
		\item $m$ is cyclic, that is, $g(m) = 1$.
	\end{enumerate} \textspace
\end{prop}
\begin{proof}
	This follows immediately by considering the generalized \hg{1} of $n$ and letting the $n_i$ be the factors corresponding to the connected component with at least two vertices,
	and $m$ the product of the values of the isolated vertices.
\end{proof}

In solving $g(n) = c$, it is useful to define the \emph{cyclic part} of $n$ to be $m$ (in this notation), and call $n$ \emph{cyclic-free} if $m = 1$.
Next, whenever $c = c_1 \cdots c_s$, finding solutions $g(n_i) = c_i$ with pairwise independent $n_i$ leads to a solution to $c$.
It is therefore natural to consider ``prime'' values of $n$, that is, those with $k = 1$.
We will call cyclic-free $n$ with $k = 1$ \emph{connected}.
By considering non-nilpotent groups, we can get a lower bound on $g(n)$.
We will only make use of the following (weak) lower bound:

\begin{prop}\thlabel{eunnp}
	If $n = \ufdsh$ is connected, then $g(n) \ge g(p_1^{\alpha_1})\cdots g(p_s^{\alpha_s}) + s - 1.$
\end{prop}

\subsection{Splicing graphs}
Given disjoint graphs $\Gamma$ and $\Lambda$ with a fixed terminal vertex $\qo$ in $\Gamma$ and any vertex $\pin$ in $\Lambda$,
we can make a new graph $\Gamma \rightarrow \Lambda$ by adjoining an arrow $\pin \rightarrow \qo$ to the union of $\Gamma$ and $\Lambda$.
This depends on the choice of $\qo$ and $\pin$ in general, but we will not explicitly refer to them when there is no risk of confusion.
In addition, we tacitly assume that we are given the functions $h_\Gamma$ and $h_\Lambda$.
We use the notation $g(X; v)$ to refer to the sum in Hölder's formula but only considering subsets $\pi$ with $v \in \pi$.

\begin{prop}\thlabel{eujoin}
	In this notation, \begin{enumrealm}\begin{equation*} g(\Gamma \rightarrow \Lambda) = g(\Gamma)g(\Lambda) + g(\Gamma - \{\qo\})g(\Lambda; \pin).
\end{equation*}\end{enumrealm}
\end{prop}
\begin{proof}
	Let $M = \Gamma \rightarrow \Lambda$.
Given a subset $\tilpi$ of $M$, we define $\piga = \pi \cap \Gamma$ and $\pila = \pi \cap \Lambda$.
For all $p \neq \qo$, we have $h_M(p, \tilpi) = h_\Gamma(p, \piga)$ if $p$ is in $\Gamma$, and $h_\Lambda(p, \pila)$ otherwise.
We have three cases:
	
	
	If $\qo \in \piga$, we have $h_M(q, \piga) = h_\Gamma(q, \piga)$ and $h_M(p, \pila) = h_\Lambda(p, \pila)$ for all $q \in \Gamma - \piga$ and $p \in \Lambda - \pila$.
It follows that $S(\tilpi, M) = S(\piga, \Gamma)S(\pila, \Lambda).$
	
	On the other hand, if $\qo\notin\piga$ and $\pin \in \pila$, we get $h_M(\qo, \tilpi) = 1$ because there is a unique edge $\qo \rightarrow \pin,$ so it contributes nothing to the product.
Since $h_M(p, \tilpi) = h_\Lambda(p, \pila)$ for all $p \in \Lambda - \pila$, we have $S(\tilpi, M) = S(\piga, \Gamma - \{\qo\})S(\pila, \Lambda)$.
	
	Finally, if $\qo \notin \piga$ and $\pin\notin\pila,$ the previous analysis shows that $h_M(\qo, \tilpi) = 0$, which in turn implies that $S(\tilpi, M) = 0$, and thus $\tilpi$ contributes nothing to the sum.
	
	The subsets $\tilpi \subseteq M$ correspond bijectively to pairs $(\piga, \pila)$ with $\piga \subseteq \Gamma$ and $\pila \subseteq \Lambda$.
From the pairs with $\qo \in \piga$ we get $g(\Gamma)g(\Lambda)$.
From the pairs with $\qo \notin \piga$, we need $\pin \in \pila,$ and the resulting terms combine to give precisely $g(\Gamma - \{\qo\})g(\Lambda; \pin).$
\end{proof}

%\setenumerate{noitemsep}
\begin{cor}\thlabel{eustick} Suppose $\Gamma$ is a \hg{1} and $v \notin \Gamma$.\listspace
	\begin{enumerate} \listspace
		\item Fix a terminal vertex $q$ in $\Gamma$.
Then $g(\Gamma \rightarrow v) = g(\Gamma) + g(\Gamma - \{q\}).$
		\item Fix an initial vertex $p$ in $\Gamma$.
Then $g(v \rightarrow \Gamma) = g(\Gamma) + g(\Gamma; p).$
	\end{enumerate}\textspace
\end{cor}

We can iterate the first operation, starting with a graph $\Gamma_0$ and a fixed terminal vertex $q$ in it and define $\Gamma_{n + 1} = \Gamma_{n} \rightarrow v_{n + 1}$, where the $v_i$ are all distinct.
If we let $\alpha = g(\Gamma_0 - \{q\})$ and $\beta = g(\Gamma_0)$, then the sequence $a_n = g(\Gamma_n)$ satisfies the recurrence relation $a_{n + 2} = a_{n + 1} + a_{n}$ with initial values $a_{-1} = \alpha$ and $a_0 = \beta$.
Starting with a single vertex, we get

\begin{prop}\thlabel{eufibo}
	Let $\Phi_n$ be a directed path of $n$ vertices.
Then $g(\Phi_n) = \fibo{n + 1}$, where $\fibo{n}$ is the Fibonacci sequence.
\end{prop}

An amusing consequence of this fact is this.
Since $\Phi_{m + n} = \Phi_m \rightarrow \Phi_n$, and we have just seen that $g(\Phi_m - \{\text{final}\}) = \fibo{m}$ and $g(\Phi_n; \text{initial}) = \fibo{n}$, 
we can now apply \hyperref[eustick]{\hthref{eustick}} to obtain the well-known identity, \begin{enumrealm}
\begin{align*}
	\fibo{m + n} &= g(\Phi_m \rightarrow \Phi_{n - 1}) \\
	&= g(\Phi_m)g(\Phi_{n - 1}) + g(\Phi_m - \{\text{final}\})g(\Phi_{n - 1}; \{\text{initial}\}) \\
	&= \fibo{m + 1}\fibo{n} + \fibo{m}\fibo{n - 1}.
\end{align*}\end{enumrealm}
\vspace{-\baselineskip}
\subsection{Some lower bounds}
We have already seen that $g(\Phi_n)$ is the Fibonacci sequence, and so it grows exponentially with $n$.
It follows that $g$ grows exponentially in the length of the longest directed path, since $g(\Gamma) \ge  g(\Gamma_0)$ for all subgraphs $\Gamma_0 \subseteq \Gamma$.
It grows exponentially in the in- and out-degrees as well:
\begin{lem}\thlabel{euinout}
	Let $r, s$ denote the in- and out-degree, respectively, of a vertex $v$ in a \hg{1} $\Gamma$ with label $p$.
Then $g(\Gamma) \ge 2^r + \frac{p^s - 1}{p - 1}$.
\end{lem}
\begin{proof}
	Suppose we have $\alpha_i \rightarrow v$ and $v \rightarrow \beta_j$ for $1 \le i \le r$ and $1 \le j \le s$, and let $\Gamma_0$ be the subgraph induced by the collection of vertices $\alpha_i$ and $\beta_j$.
Evidently, any subset of $\Gamma_0$ containing $v$ and the $\beta_j$ is central, and there are $2^r$ such subsets.
Moreover, the complement of $\{v\}$ in $\Gamma_0$ is central, and it contributes $\frac{p^s - 1}{p - 1}$.
This shows that $\Gamma_0$, and thus $\Gamma$, satisfies the inequality.
\end{proof}

% In the sequel, we consider generalized \hgs{1} of cube-free numbers.
%There are two groups of order $p^2$, namely $\cyc{p}^2 = \cyc{p} \times \cyc{p}$ and $\cyc{p^2}$.
%In both cases, we have an epimorphism $P \twoheadrightarrow \cyc{p}$, and thus from each semidirect product $G \rtimes \cyc{p}$, we get \textit{two} semidirect products $G \rtimes P$.
%On the other hand, we have a monomorphism $\aut{\cyc{p}} \hookrightarrow \aut{P}$ in both cases: For $P = \cyc{p} \times \cyc{p}$ by fixing one factor and acting on the other, and for $P = \cyc{p^2}$ from the well-known fact that $\aut{P} \cong \cyc{p} \times \cyc{p - 1}$.
%It follows that the same observation holds for $P \rtimes G$ as well.
%These two facts together imply the following
% \begin{lem}\thlabel{eubold}
	% If $n$ is cube-free with $s$ factors of the form $p^2$, then $g(n) \ge 2^s g(\operatorname{rad}(n))$.
% \end{lem}
% \begin{proof}
	% For a group $G$ of order $n_0 = \operatorname{rad}(n)$ and a subset $\tau \subset \sigma$, where $\sigma$ is the set of prime-squared factors of $n$, it can be seen that we can replace the $p$-Sylow subgroup of each $p$ in $\tau$ with either $\cyc{p} \times \cyc{p}$ or $\cyc{p}^2$, essentially since $G$ is an iterated semidirect product.
% Thus we can form at least $2^s$ different groups 
% \end{proof}